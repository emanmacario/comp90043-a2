    
\documentclass[12pt]{article}
\usepackage{graphicx}
\usepackage{xspace}
\usepackage{color, colortbl}
\usepackage{amssymb}
\usepackage{amsmath}
\usepackage{mathtools}
\usepackage{listings}
\pagestyle{empty}
\definecolor{Gray}{gray}{0.9}
\textwidth      165mm
\textheight     240mm
\topmargin      -18mm
\oddsidemargin  -2mm
\evensidemargin 2mm
\newcommand{\impl}{\mathbin{\Rightarrow}}
\newcommand{\biim}{\mathbin{\Leftrightarrow}}
\renewcommand{\theenumi}{\alph{enumi}}


\DeclarePairedDelimiter\Floor\lfloor\rfloor
\DeclarePairedDelimiter\Ceil\lceil\rceil

\author{Emmanuel Macario - 831659}
\title{COMP90043 Cryptography \& Security Assignment 2}
\date{Semster 2, 2019}

\begin{document}
\maketitle

% Question 1
\subsection*{Question 1}

\begin{enumerate}
\item Lianglu could recover the message without knowing Ge's or Jiajia's private key by considering the following:

Since different $e$ are relatively prime, then $gcd(e_{G}, e_{J})=1$. Hence, $\exists x, y \in \mathbb{Z}$ such 
that $xe_{G}+ye_{J}=1$. We can use XGCD to find $x, y$. Then, we can use $x,y$ to find $M$ as follows:


\begin{center}
$C_{G}^{x} C_{J}^{y} = (M^{e_{G}})^{x} (M^{e_{J}})^{y}$
$                                  = M^{{x e_{G}}+{y e_{J}}}$
$                                  = M^{1}$
$                                  = M \text{ (mod }n)$

\end{center}

\item Lianglu can recover Zhuohan's private key using the following strategy.

Firstly, since we know $d_{G}$, we can factor $n$. Then, we can use XGCD to obtain $d_{Z}$.

In order to factor $n$, let $k=d_{G}e - 1$. Since $ed-1 = 0$ mod $\phi (n)$, then $k$ is a multiple of $\phi (n)$. 
We also know that $k$ must be even. Hence, let $k=2^{t}r$ with $r$ odd and $t \geq 1$. Now, we pick a random
generator $g$, where $1 < g < N$, and compute the sequence $g^{k/2}, g^{k/4}, \dots, g^{k/2^{t}}$. We determine
the first sequence element $x \neq \pm 1$, where $gcd(x-1, n) > 1$.

If no such element exists, then choose another $g$ and try again.

Otherwise, let $p = gcd(x-1, n)$ and $q = n/p$.


Now that we have $p$ and $q$, we can compute $\phi (n) = (p-1)(q-1)$. Using XGCD, we can now find
$d_{Z} = e_{Z}^{-1}$ mod $\phi (n)$.


\end{enumerate}


% Question 2
\subsection*{Question 2}
\begin{enumerate}

\item Encryption function for Ge:
\begin{center}
$E_{G}(M) = M^{k_{1}}$ mod $n$
\end{center}

Decryption function for Jiajia:
\begin{center}
$D_{J}(C) = C^{k_{2} k_{3}}$ mod $n$
\end{center}

\item Encryption function for Jiajia: 
\begin{center}
\[
  E_{J \rightarrow X}(M)=
  \begin{cases}
  	M^{k_{3}} \text{ mod } n & \text{if $X=$ Ge} \\
  	M^{k_{2}} \text{ mod } n & \text{if $X=$ Zhuohan}
  \end{cases}
\]
\end{center}

Decryption function for Ge:
\begin{center}
$D_{G}(C) = C^{k_{1} k_{2}}$ mod $n$
\end{center}

Decryption function for Zhuohan:
\begin{center}
$D_{Z}(C) = C^{k_{1} k_{3}}$ mod $n$
\end{center}


\end{enumerate}

% Question 3
\subsection*{Question 3}

\begin{enumerate}
\item (1) A sends a request message for a new session key to B, which includes the identity of A, and a unique nonce $N_{a}$
encrypted with their secret key $K_{a}$

(2) B sends A's original message, plus their own message mirroring A's, which includes the identity of B,
and a unique nonce $N_{b}$ encrypted with their secret key $K_{b}$, to the KDC

(3) The KDC sends B a response message. The message includes two components, one meant for A, and the other meant for B.
Each component is encrypted with the intended recipient's secret key, and includes:
\begin{itemize}
	\item The one time session key, $K_{s}$, to be used for the session
	\item The identity of the opposite communicating user (e.g. $ID_{a}$ for B)
	\item The original nonce, to enable the users to match the response with the corresponding request
\end{itemize}

(4) B forwards A's component, which they received from the KDC, to A. Now, both A and B have access to the session key.


\item To estimate the storage requirements of both the KDC and users, some assumptions must be made.
We assume that a session key is discarded after use, and that the KDC does not store session keys. For each
user, there is a master key that they share with the KDC. Hence, if there are a total of $N$ unique users, then
the KDC is required to persistently store $N$ master keys, one for each user.

In contrast, each user must only store their own master key. Session keys must also be kept for the
duration of the corresponding session.

\item This scheme does not ensure the authenticity of both A and B. The authenticity of B can be compromised if
either B or an attacker C sends the KDC the message $ID_{c} \Vert E(K_{c}, N_{c})$ (or any message consisting of a
matching identity and encrypted nonce pair). This means that when A receives a response, the identity of the 
responder will not be who they believed it to be.

\item A replay attack can be detected in this scheme, and can be identified in steps (2), (3), and (4). In step (2), if the KDC
is sent a message $ID_{x} \Vert E(K_{y}, N_{y})$, where $x \neq y$, then the KDC may be suspicious of a replay
attack, since they cannot decrypt the encrypted nonce. In both steps (3) and (4), both B and A can detect a replay
attack if the nonce that they receive is not equivalent to their original nonce.

\end{enumerate}

% Question 4
\subsection*{Question 4}
\begin{enumerate}
\item Yes. Each user must have access to the other user's public key. The procedure can be summarised in the following three steps:
\begin{center}
(1) Alice $\rightarrow$ Bob: $g^{x}$

(2) Alice $\leftarrow$ Bob: $g^{y}, E_{g^{xy}}(S_{B}(g^{y}, g^{x}))$

(3) Alice $\rightarrow$ Bob: $E_{g^{xy}}(S_{A}(g^{x}, g^{y}))$
\end{center}

\item No. An attacker can simply change the data stream and recompute the hash value, before sending the message.

\item Yes. The procedure can be summarised in the following four steps:
\begin{center}
(1) Alice $\rightarrow$ Bob: $g^{x}, C(K_{AB}, g^{x})$

(2) Alice $\leftarrow$ Bob: $g^{y}, C(K_{AB}, g^{y})$

(3) Alice: Test if received MAC matches calculated MAC

(4) Bob: Test if received MAC matches calculated MAC
\end{center}
\end{enumerate}


% Question 5
\subsection*{Question 5}
\begin{enumerate}
\item Yes. If the last block $M_{i}$ has length less than the fixed size, then it is possible to
pad it to the specified fixed length.

\item Yes, since the output is always mod $n$.

\item No, since a message containing many blocks may require numerous expensive exponentiation computations.

\item Yes, since the hash function is based on the RSA problem.

\item No.

\item No.
\end{enumerate}
\end{document}