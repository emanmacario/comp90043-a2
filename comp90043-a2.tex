    
\documentclass[12pt]{article}
\usepackage{graphicx}
\usepackage{xspace}
\usepackage{color, colortbl}
\usepackage{amssymb}
\usepackage{amsmath}
\usepackage{mathtools}
\usepackage{listings}
\pagestyle{empty}
\definecolor{Gray}{gray}{0.9}
\textwidth      165mm
\textheight     240mm
\topmargin      -18mm
\oddsidemargin  -2mm
\evensidemargin 2mm
\newcommand{\impl}{\mathbin{\Rightarrow}}
\newcommand{\biim}{\mathbin{\Leftrightarrow}}
\renewcommand{\theenumi}{\alph{enumi}}


\DeclarePairedDelimiter\Floor\lfloor\rfloor
\DeclarePairedDelimiter\Ceil\lceil\rceil

\author{Emmanuel Macario - 831659}
\title{COMP90043 Cryptography \& Security Assignment 2}
\date{Semster 2, 2019}

\begin{document}
\maketitle

% Question 1
\subsection*{Question 1}

\begin{enumerate}
\item Lianglu could recover the message without knowing Ge's or Jiajia's private key by the following:

Since different $e$ are relatively prime, then $gcd(e_{G}, e_{J})=1$. Hence, $\exists x, y \in \mathbb{Z}$ such 
that $xe_{G}+ye_{J}=1$. We can use XGCD to find $x, y$. Then, we have


\begin{center}
$C_{G}^{x} C_{J}^{y} = (M^{e_{G}})^{x} (M^{e_{J}})^{y}$
$                                  =  M^{{x e_{G}}^+{y e_{J}}}$
$                                  = M^{1}$
$                                  = M (mod n)$

\end{center}

\item Lianglu can recover Zhuohan's private key using the following strategy.

Given $d_{G}$, we can factor $n$ using the following method, then use XGCD to obtain $d_{Z}$.

Firstly, we compute $k=d_{G}e - 1$. Since $ed$ mod $\phi (n) = 1$, then $k$ is a multiple of $\phi (n)$. 
We also know that $k$ must be even. Hence, let $k=2^{t}r$ with $r$ odd and $t \geq 1$. Now, we pick a random
generator $g$, where $1 < g < N$, and compute the sequence $g^{k/2}, g^{k/4}, \dots, g^{k/2^{t}}$. We determine
the first sequence element $g^{x} \neq \pm 1$, where the previous element equals {1}.

If no such element exists, then choose another $g$ and try again.

Otherwise, let $p = gcd(g^{x}-1, n)$ and $q = n/p$.


Now that we have $p, q$, we can compute $\phi (n) = (p-1)(q-1)$. Using XGCD, we can now find
$d_{Z} = e_{Z}^{-1}$ mod $\phi (n)$.


\end{enumerate}


% Question 2
\subsection*{Question 2}
\begin{enumerate}

\item Encryption function for Ge:
\begin{center}
$E_{G}(M) = M^{k_{1}}$ mod $n$
\end{center}

Decryption function for Jiajia:
\begin{center}
$D_{J}(C) = C^{k_{2} k_{3}}$ mod $n$
\end{center}

\item Encryption function for Jiajia: 
\begin{center}
\[
  E_{J \rightarrow X}(M)=
  \begin{cases}
  	M^{k_{3}} \text{ mod } n & \text{if $X=$ Ge} \\
  	M^{k_{2}} \text{ mod } n & \text{if $X=$ Zhuohan}
  \end{cases}
\]
\end{center}

Decryption function for Ge:
\begin{center}
$D_{G}(C) = C^{k_{1} k_{2}}$ mod $n$
\end{center}

Decryption function for Zhuohan:
\begin{center}
$D_{Z}(C) = C^{k_{1} k_{3}}$ mod $n$
\end{center}


\end{enumerate}

% Question 3
\subsection*{Question 3}

\begin{enumerate}
\item A

\item B

\item C

\item D
\end{enumerate}

% Question 4
\subsection*{Question 4}
\begin{enumerate}
\item A

\item B
\end{enumerate}

\end{document}